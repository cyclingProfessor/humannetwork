\documentclass[svgnames]{beamer}
\mode<presentation>
{
%  \usetheme{Boadilla}
  \usetheme{Madrid}
}

\usepackage{pgfpages}
\usepackage[tikz]{bclogo}
\usepackage{colortbl}
\newcolumntype{b}{>{\columncolor{yellow}}c}
\newcolumntype{a}{>{\columncolor{orange}}c}
\newcolumntype{d}{>{\color{white}\columncolor{blue}}c}
\newcolumntype{e}{>{\columncolor{purple}}c}

\usepackage[font={sc,color=red},labelformat=empty]{caption}

\usepackage[T1]{fontenc}
\usepackage{frcursive}
%\usepackage{calligra}
\usetikzlibrary{positioning}
\usepackage[english]{babel}
%\usepackage[latin1]{inputenc}
\usepackage{times}
\usepackage{alltt}

\newcommand{\setfont}[2]{{\fontfamily{#1}\selectfont #2}}
\mode<handout>
{
\pgfpagesuselayout{4 on 1}[a4paper,landscape,border shrink=3mm]
}



%\usepackage[T1]{fontenc}

\title{Build Your Own Internet}

\subtitle{Be Part of the Solution}

\author % (optional, use only with lots of authors)
{The Outreach Team}
%\date{February 13th, 2014}
\date{}

\institute[RHUL] % (optional, but mostly needed)
{
  Department of Computer Science\\
  Royal Holloway College\\
  University of London
}

\begin{document}

\begin{frame}
  \titlepage
\end{frame}

\begin{frame}{The Setup}
\begin{alertblock}{Zombie Apocalypse: A credible threat?
}
\begin{center}

We cannot afford not to know.  

The internet will fail.  

How can we build a Human Internet!
\end{center}
\end{alertblock}
\end{frame}
\begin{frame}{The Game}
\begin{block}{Process}
\begin{itemize}
\item You are facing a screen.
\item Check that you are logged in and can see the web browser open at \url{http://cs.rhul.io/byoi}
\item \dots then wait\ldots (more on the next slide)
\end{itemize}
\end{block}
\end{frame}
\begin{frame}{The Software}
\begin{block}{You are a node in a network connected to other nodes}
\begin{itemize}
\item You can send/receive messages \textbf{only} to nodes to which you are \alert{directly} connected.
\item You "broadcast" to all \alert{directly} connected nodes.
\item Sending to a node to which you are not \alert{directly} connected will \textbf{silently} fail.
\item Your screen will have several TABS (list of messages) - all sent and received messages will appear in these lists.
\item Only one TAB shows \alert{all} of the messages that you send/receive.
\item You have a box in which you can type a message, and buttons that process messages.
\item \alert{Try sending a message now\ldots}
\end{itemize}
\end{block}
\end{frame}

\begin{frame}{Task: Maintain the Network (RIP, broadcast)}
\vspace*{-2mm}
\begin{alertblock}{Find all the connections in your network.}
\begin{itemize}
\item \alert{There is no talking required in this game}.
\item Just edit and send messages from your interface.
\item You are \alert{not all in one network}.
\item \large You have until the first person correctly knows their network.
\end{itemize}
\end{alertblock}

\vspace*{-2mm}
\begin{block}{Your Interface: Your own node number is at the top.}
\begin{itemize}
\item One TAB shows you all messages that you have sent or received.
\item Each node you receive a message from gets its own TAB.
\item Selecting a message from any message list enters it into the sending box.
\item Sending to Node 0 broadcasts to all \alert{directly} connected nodes.
\end{itemize}
\end{block}
\pause
\begin{alertblock}{Hint. The Internet is a cooperative game.}
{Broadcast your knowledge as lists: e.g. 17-23-45-2-67-4}
\end{alertblock}
\end{frame}

\begin{frame}{Task: Send a long Message (fragmentation)}
\begin{block}{Your Interface}
\begin{itemize}
\item Your Task is on the top of your screen.
\item You will send a (long) message to send to another node.
\item You will need to use the \alert{Split} button to break up long text.
\item The \alert{Split} button will add fragment IDs (\alert{"n/m:"}) to each fragment.
\item The received message can be reassembled by selecting all parts and pressing \alert{Merge}.
%\item Three points for every delivered (complete) message.
\end{itemize}
\end{block}
\end{frame}

\begin{frame}{Task: Multi-Hop a long Message (forwarding)}
\begin{block}{Your Interface}
\begin{itemize}
\item Your task is at the top of your screen.
\item You will send a (long) message to send to another node.
\item \dots but you are not connected to that node.
\item A \alert{!!} in the message splits a message header from the body.
\item Message headers are copied to every fragment when you \alert{split}.
\item Use a header that gives intermediate nodes enough information to forward your fragments.
\end{itemize}
\end{block}
\end{frame}

\begin{frame}{Task: Multi-Hop on a Typical Network (acknowledge)}
\begin{block}{Your Interface}
\begin{itemize}
\item Your task is at the top of your screen.
\item You will send a (long) message to send to another node.
\item \dots but you are not connected to that node.
\item \dots and some messages will be \alert{silently} dropped, or delayed.
\item Your screen indicates the percentage of messages that will not be delivered.
\item How will you be able to tell if a message has been delivered?
\end{itemize}
\end{block}
\pause
\begin{alertblock}{Hint. Acknowledge with enough information.}
{Simply send back each received message as an acknowledgement.}
\end{alertblock}

\end{frame}
\begin{frame}{Task: A Malicious Intruder (encryption, checksums, nonces)}
\begin{block}{Your Interface}
\begin{itemize}
\item Your task is at the top of your screen.
\item You will send a (long) message to send to another node.
\item \dots but you are not connected to that node.
\item \dots and some messages will be \alert{silently} dropped, or delayed.
\item \dots and some messages will be \alert{altered/corrupted by the network}
\item Your screen indicates the percentage of messages that will not be delivered, or will be corrupted.
\item You will need to \alert{encrypt} and \alert{checksum} each message so that it cannot be altered.
\item Why not add a \alert{nonce} so that malicious intruders cannot replay old messages?
\end{itemize}
\end{block}
\end{frame}
\begin{frame}{Task: Protocols}
\begin{block}{WHOIS Protocol}
\begin{itemize}
\item All messages in this round must be one of:
\begin{enumerate}
\item \texttt{WHOIS(Query, Node nnn)} where \texttt{nnn} is a number
\item \texttt{WHOIS(Answer, Node nnn is name)} where \texttt{nnn} is the node number of \texttt{name}
\end{enumerate}
\item You can use a paper and pencil to make notes.
\item You have twenty minutes.
\item One point for every name discovered.
\end{itemize}
\end{block}
\end{frame}

\end{document}
